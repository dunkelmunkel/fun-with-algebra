\documentclass{../cssheet}


%--------------------------------------------------------------------------------------------------------------
% Basic meta data
%--------------------------------------------------------------------------------------------------------------

\title{Abstraktion --- Präsenzaufgaben}
\author{Prof. Dr. Christian Spannagel}
\date{\today}
\hypersetup{%
    pdfauthor={\theauthor},%
    pdftitle={\thetitle},% 
    pdfsubject={Aufgabenblatt Algebra},%
    pdfkeywords={algebra}
}

%--------------------------------------------------------------------------------------------------------------
% document
%--------------------------------------------------------------------------------------------------------------

\begin{document}
\printtitle


\textbf{Aufgabe 1 (Natürliche Zahlen als Abstraktion):}
In der Mathematik begegnen uns natürliche Zahlen ständig – aber was genau ist eigentlich eine natürliche Zahl?

\begin{enumerate}[a)]
\item Sammelt gemeinsam verschiedene Situationen aus dem Alltag, in denen natürliche Zahlen verwendet werden (z.\,B. beim Zählen von Personen, beim Ordnen in einer Warteschlange, \ldots).
\item Welche unterschiedlichen Rollen spielen natürliche Zahlen in euren Beispielen? Welche \glqq{}Bedeutung\grqq{} haben sie jeweils?
\item Versucht nun, diese Rollen zu benennen. Wenn euch Begriffe fehlen, beschreibt sie zunächst in eigenen Worten!
\item Überlegt: Warum kann man sagen, dass die natürlichen Zahlen eine Abstraktion sind? Wovon wird abstrahiert?
\end{enumerate}

\textbf{Aufgabe 2 (Abstraktion bei Zahlbereichen):}  Betrachtet in dieser Aufgabe einmal die Zahlenmengen  $\mathbb{N}$, $\mathbb{Z}$, $\mathbb{Q}$,  $\mathbb{R}$ und  $\mathbb{C}$ mit den üblichen Rechenoperationen.

\begin{enumerate}[a)]
\item Beschreibt, welche Zahlen in den jeweiligen Mengen enthalten sind. Wofür stehen die Symbole?
\item Welche Eigenschaften haben die Zahlenmengen, auch in Verbindung mit den Rechenoperationen? Welche Rechenregeln gelten?
\item Wodurch unterscheiden sich die Zahlenmengen?
\item Welche Gleichungen kann man in den jeweiligen Zahlenmengen lösen, welche nicht?
\item Abstraktionsschritt: Was haben einige oder alle dieser Zahlenmengen gemeinsam? Welche abstrakten Strukturen lassen sich erkennen?
\end{enumerate}


\textbf{Aufgabe 3 (Weitere Abstraktionen):}  Schaut euch einmal in anderen Teilgebieten der Mathematik um (Geometrie, Analysis, Stochastik, \ldots). Findet unterschiedliche mathematische Konzepte / Strukturen /  Objekte / \ldots, die gleiche Eigenschaften haben und/oder die gleiche Operationen ermöglichen. Wie kann man hier abstrahieren?

\vspace*{2cm}
\printlicense

\printsocials


\end{document}
