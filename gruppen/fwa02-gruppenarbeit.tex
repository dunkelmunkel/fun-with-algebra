\documentclass{../cssheet}

%--------------------------------------------------------------------------------------------------------------
% Basic meta data
%--------------------------------------------------------------------------------------------------------------

\title{Gruppenarbeit}
\author{Prof. Dr. Christian Spannagel}
\date{\today}
\hypersetup{%
    pdfauthor={\theauthor},%
    pdftitle={\thetitle},% 
    pdfsubject={Aufgabenblatt Algebra},%
    pdfkeywords={algebra}
}

%--------------------------------------------------------------------------------------------------------------
% document
%--------------------------------------------------------------------------------------------------------------

\begin{document}
\printtitle


\textbf{Aufgabe 1 (Gruppensuchen Teil 2):} Welche der folgenden Strukturen sind eine Gruppe und warum? Welche von den Gruppen sind abelsch? Und falls eine Struktur keine Gruppe ist: Ist sie vielleicht eine Halbgruppe?

\begin{enumerate}[a)]
\item Die Menge aller $2\times 2$-Matrizen reeller Zahlen mit der Matrixaddition als Verknüpfung
\item Die Menge aller $2\times 2$-Matrizen reeller Zahlen mit der Matrixmultiplikation als Verknüpfung
\item Die Menge aller linearen reellen Funktionen mit der Verkettung als Verknüpfung:\\ $(f\circ g) (x) = f ( g (x))$
\item Die Menge aller n-stelligen Permutationen mit der Verkettung als Verknüpfung. Beispiel (von rechts nach links denken!):\\
\\ 
\begin{math}\begin{pmatrix} 1 & 2 & 3 \\ 3 & 1 & 2 \end{pmatrix} \circ \begin{pmatrix} 1 & 2 & 3 \\ 2 & 1 & 3 \end{pmatrix} = \begin{pmatrix} 1 & 2 & 3 \\ 1 & 3 & 2 \end{pmatrix}\end{math}
\end{enumerate}


\vspace*{2cm}
\printlicense

\printsocials

\end{document}
