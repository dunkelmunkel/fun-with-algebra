\documentclass{../cssheet}


%--------------------------------------------------------------------------------------------------------------
% Basic meta data
%--------------------------------------------------------------------------------------------------------------

\title{Gruppen und Halbgruppen}
\author{Prof. Dr. Christian Spannagel}
\date{\today}
\hypersetup{%
    pdfauthor={\theauthor},%
    pdftitle={\thetitle},% 
    pdfsubject={Aufgabenblatt Algebra},%
    pdfkeywords={algebra}
}

%--------------------------------------------------------------------------------------------------------------
% document
%--------------------------------------------------------------------------------------------------------------

\begin{document}
\printtitle

\vspace*{10mm}

\textbf{Aufgabe 1 (Was ist eine Gruppe?):} 
\begin{enumerate}[a)]
\item Informiert euch im Internet: Was ist eine Gruppe? Schreibt euch die Definition einer Gruppe auf. 
\item Was ist eine abelsche Gruppe?
\item Was ist eine Halbgruppe?
\item Was ist die \emph{Ordnung} einer Gruppe?
\end{enumerate}

\textbf{Aufgabe 2 (Gruppensuchen):} Welche der folgenden Strukturen sind eine Gruppe und warum? Welche von den Gruppen sind abelsch?

\begin{enumerate}[a)]
\item $(\mathbb{N}; +)$ \emph{(Hinweis: In dieser Veranstaltung ist die $0$ Element der natürlichen Zahlen.)}
\item $(\mathbb{N}; \cdot)$
\item $(\mathbb{Z}; +)$
\item $(\mathbb{Z}; \cdot)$
\item $(2\mathbb{Z}; +)$ \emph{(Hinweis: $2\mathbb{Z}$ ist die Schreibweise für die Menge der geraden ganzen Zahlen)}
\item $(\mathbb{Q}; +)$
\item $(\mathbb{Q}; \cdot)$
\item $(\mathbb{Q}^{+}; \cdot)$
\item $(\mathbb{R}; \cdot)$
\item $(\mathbb{R}\setminus\{0\}; \cdot)$
\end{enumerate}

\textbf{Aufgabe 3 (Keine halben Sachen):} Welche der Strukturen in Aufgabe~2 sind Halbgruppen und warum?


\vspace*{2cm}
\printlicense

\printsocials


\end{document}
