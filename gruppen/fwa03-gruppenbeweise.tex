\documentclass{../cssheet}


%--------------------------------------------------------------------------------------------------------------
% Basic meta data
%--------------------------------------------------------------------------------------------------------------

\title{Gruppenbeweise}
\author{Prof. Dr. Christian Spannagel}
\date{\today}
\hypersetup{%
    pdfauthor={\theauthor},%
    pdftitle={\thetitle},% 
    pdfsubject={Aufgabenblatt Algebra},%
    pdfkeywords={algebra}
}

%--------------------------------------------------------------------------------------------------------------
% document
%--------------------------------------------------------------------------------------------------------------

\begin{document}
\printtitle

Das Praktische an Strukturen wie Gruppen ist: Hat man etwas für Gruppen im Allgemeinen bewiesen, dann gilt es für jede Struktur, die eine Gruppe ist!

\textbf{Aufgabe 1 (Highlander):}  Beweist folgende Aussagen:
\begin{enumerate}[a)]
\item In jeder Gruppe existiert \emph{genau ein} Neutralelement.
\item In jeder Gruppe existiert zu jedem Element \emph{genau ein} Inverses.
\end{enumerate}

\textbf{Aufgabe 2 (Wir vereinfachen uns das Leben):}  In der Gruppendefinition wird gefordert, dass das Neutralelement sowohl links- als auch rechtsneutral ist. Beweist, dass aus der Linksneutralität des Neutralelemtents automatisch dessen Rechtsneutralität folgt. Zeigt dasselbe auch für die Inversen: Wenn ein Element linksinvers ist, dann ist es automatisch auch rechtsinvers. (Warum erleichtert uns das in Zukunft die Arbeit?)

\textbf{Aufgabe 3 (Praktisches):}  Beweist die folgenden Aussagen:
\begin{enumerate}[a)]

\item In jeder Gruppe $(G; \cdot)$ besitzt jede Gleichung der Form $a \cdot x = b$ und $y \cdot a = b$ \emph{genau eine} Lösung.
\item Für alle Elemente einer Gruppe $a$ und $b$ gilt die \emph{Socke-Schuh-Regel} bzw. \emph{Hemd-Jacken-Regel}:\\ $(a \cdot b)^{-1}=b^{-1}\cdot a^{-1}$
\end{enumerate}

\vspace*{2cm}
\printlicense

\printsocials

\end{document}
