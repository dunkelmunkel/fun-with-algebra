\documentclass{cssheet}


%--------------------------------------------------------------------------------------------------------------
% Basic meta data
%--------------------------------------------------------------------------------------------------------------

\title{Endliche Gruppen}
\author{Prof. Dr. Christian Spannagel}
\date{\today}
\hypersetup{%
    pdfauthor={\theauthor},%
    pdftitle={\thetitle},% 
    pdfsubject={Aufgabenblatt Algebra},%
    pdfkeywords={algebra}
}

%--------------------------------------------------------------------------------------------------------------
% document
%--------------------------------------------------------------------------------------------------------------

\begin{document}
\printtitle


\textbf{Aufgabe 1 (Verknüpfungstafel):} 

Gegeben sei die Struktur  $(\{e,r,r^2,s,sr,sr^2\};\circ)$  mit folgender Verknüpfungstafel:

\[
\begin{array}{c|cccccc}
	\circ & e & r & r^2 & s & sr & sr^2 \\
	\hline
	e     & e & r & r^2 & s & sr & sr^2 \\
	r     & r & r^2 & e & sr & sr^2 & s \\
	r^2   & r^2 & e & r & sr^2 & s & sr \\
	s     & s & sr^2 & sr & e & r^2 & r \\
	sr    & sr & s & sr^2 & r & e & r^2 \\
	sr^2  & sr^2 & sr & s & r^2 & r & e \\
\end{array}
\]


\begin{enumerate}[a)]
\item Handelt es sich um eine Gruppe?
\item Welche geometrische Interpretation könnt ihr finden?
\item Eine Untergruppe ist eine Struktur, bestehend aus einer Teilmenge der Trägermenge der gegebenen Gruppe mit derselben Operation. Welche Untergruppen könnt ihr finden?
\end{enumerate}



\vspace*{2cm}
\printlicense

\printsocials

\end{document}

\end{document}
