\documentclass{../cssheet}


%--------------------------------------------------------------------------------------------------------------
% Basic meta data
%--------------------------------------------------------------------------------------------------------------

\title{Endliche Gruppen}
\author{Prof. Dr. Christian Spannagel}
\date{\today}
\hypersetup{%
    pdfauthor={\theauthor},%
    pdftitle={\thetitle},% 
    pdfsubject={Aufgabenblatt Algebra},%
    pdfkeywords={algebra}
}

%--------------------------------------------------------------------------------------------------------------
% document
%--------------------------------------------------------------------------------------------------------------

\begin{document}
\printtitle


\textbf{Aufgabe 1 (Verknüpfungstafeln):} 

Stellt die Verknüpfungstafeln für folgende Strukturen auf (die angegebenen Mengen jeweils mit der Verkettung als Verknüpfung). Wie viele Elemente hat die jeweilige Menge? Handelt es sich bei den Strukturen jeweils um Gruppen? Argumentiert!
\begin{enumerate}[a)]
\item Die Menge der Deckabbildungen des gleichschenkligen Dreiecks
\item Die Menge der Drehsymmetrien des gleichschenkligen Dreiecks
\item Die Menge der Achsensymmetrien des gleichschenkligen Dreiecks
\item Die Menge der Deckabbildungen des Rechtecks
\item Die Menge der Drehsymmetrien des Quadrats
\end{enumerate}

\textbf{Aufgabe 2 (Mini-Gruppen):}  Gibt es Gruppen mit nur einem oder nur zwei Elementen? Wenn ja: Stellt die Verknüpfungstafeln auf und begründet, dass es Gruppen sind.


\vspace*{2cm}
\printlicense

\printsocials

\end{document}

\end{document}
