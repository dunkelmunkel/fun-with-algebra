\documentclass{../cssheet}


%--------------------------------------------------------------------------------------------------------------
% Basic meta data
%--------------------------------------------------------------------------------------------------------------

\title{Verknüpfungstafeln}
\author{Prof. Dr. Christian Spannagel}
\date{\today}
\hypersetup{%
    pdfauthor={\theauthor},%
    pdftitle={\thetitle},% 
    pdfsubject={Aufgabenblatt Algebra},%
    pdfkeywords={algebra}
}

%--------------------------------------------------------------------------------------------------------------
% document
%--------------------------------------------------------------------------------------------------------------

\begin{document}
\printtitle


\textbf{Aufgabe 1 (Gruppentest mit Verknüpfungstafel):} 

In der Struktur $(\{e,a,b,c,d\};\circ)$ ist die Operation $\circ$ durch eine Verknüpfungstafel gegeben. Handelt es sich um eine Gruppe?

\[
\begin{array}{c|ccccc}
\circ & e & a & b & c & d \\
\hline
e & e & a & b & c & d \\
a & a & e & c & d & b \\
b & b & d & e & a & c \\
c & c & b & d & e & a \\
d & d & c & a & b & e \\
\end{array}
\]

\textbf{Aufgabe 2 (Vervollständigung):} 

Vervollständigt die folgende Verknüpfungstafel so, dass $(\{e,a,b,c,d\};\circ)$ eine Gruppe ist.

\[
\begin{array}{c|ccccc}
\circ & e & a & b & c & d \\
\hline
e &  &  &  & c &  \\
a &  & b & c &  &  \\
b &  &  &  &  &  \\
c &  &  &  &  &  \\
d &  &  &  &  &  \\
\end{array}
\]

\vspace*{2cm}
Die Aufgaben orientieren sich an Göthner, P. (1997). \emph{Elemente der Algebra}. Stuttgart, Leipzig: Teubner. S.~21

\vspace*{2cm}
\printlicense

\printsocials

\end{document}

\end{document}
